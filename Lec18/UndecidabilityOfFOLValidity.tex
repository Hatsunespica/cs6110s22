% --------------------------------------------------------------
% This is all preamble stuff that you don't have to worry about.
% Head down to where it says "Start here"
% --------------------------------------------------------------

\documentclass[11pt]{article}


\usepackage{graphicx,url}
\usepackage{proof}
\usepackage{framed}
\usepackage{etaremune}

\usepackage[margin=1in]{geometry}
\usepackage{amsmath,amsthm,amssymb,amsfonts}
\usepackage{paralist}

\usepackage[most]{tcolorbox}

\definecolor{bg}{rgb}{0.9,0.9,0.9}%-- try others later
\definecolor{lightgray}{rgb}{0.95,0.95,0.95}%-- try others later
\definecolor{block-gray}{gray}{0.9}%-- try others later
\definecolor{dark-gray}{gray}{0.86} %-- try others later
\definecolor{light-gray}{gray}{0.96} %-- try others later
\newtcolorbox{myquote}{colback=block-gray,breakable,boxrule=0pt}
\newtcolorbox{mydarkquote}{colback=dark-gray,breakable,boxrule=0pt}
\newtcolorbox{mylightquote}{colback=light-gray,breakable,boxrule=0pt}


% 1. To get version suitable for students to populate,
%    remove the contents of the \ignoreSoln{..body..}
%
% 2. To get a version suitable for generating PDF 
%    without solutions, remove the #1 below
%
% 3. To generate solutions, keep the #1 below
%
% 4. Assigned grader fills \ignoreSoln{..body..}
%    and also provides his/her feedback to student
%    and policy followed for point deduction
%    So design policy before grading begins.

\newcommand{\ignoreSoln}[1]{#1}   
%\newcommand{\ignoreModel}[1]{#1} 


\newcommand{\bigset}[2]{\big\{\;#1\;:\;#2\;\big\}}
\newcommand{\N}{\mathbb{N}}
\newcommand{\Z}{\mathbb{Z}}
\newcommand{\R}{\mathbb{R}}
\newcommand{\Np}{\mathbb{N^{+}}}

\newenvironment{theorem}[2][Theorem]{\begin{trivlist}
\item[\hskip \labelsep {\bfseries #1}\hskip \labelsep {\bfseries #2.}]}{\end{trivlist}}
\newenvironment{lemma}[2][Lemma]{\begin{trivlist}
\item[\hskip \labelsep {\bfseries #1}\hskip \labelsep {\bfseries #2.}]}{\end{trivlist}}
\newenvironment{exercise}[2][Exercise]{\begin{trivlist}
\item[\hskip \labelsep {\bfseries #1}\hskip \labelsep {\bfseries #2.}]}{\end{trivlist}}
\newenvironment{reflection}[2][Reflection]{\begin{trivlist}
\item[\hskip \labelsep {\bfseries #1}\hskip \labelsep {\bfseries #2.}]}{\end{trivlist}}
\newenvironment{proposition}[2][Proposition]{\begin{trivlist}
\item[\hskip \labelsep {\bfseries #1}\hskip \labelsep {\bfseries #2.}]}{\end{trivlist}}
\newenvironment{corollary}[2][Corollary]{\begin{trivlist}
\item[\hskip \labelsep {\bfseries #1}\hskip \labelsep {\bfseries #2.}]}{\end{trivlist}}

\DeclareMathSizes{14}{14}{14}{14}

\begin{document}

% --------------------------------------------------------------
%                         Start here
% --------------------------------------------------------------

%\renewcommand{\qedsymbol}{\filledbox}


\begin{center}
\begin{Large}
\sf Undecidability of the Validity FOL -- presented as steps we one can check off
\end{Large}
\end{center}

\begin{Large}

\noindent These are meant as steps you can check off pertaining to the proof of validity of FOL. Each step
later will refer to a previous step, so you can monitor where you don't follow something.
%
Check each box as you read through.



\begin{enumerate}
\item \verb|[ ]| \label{step1}
  A PCP instance $S = \{(\alpha_1, \beta_1), (\alpha_2, \beta_2), \ldots, (\alpha_n, \beta_n) \}$ is given.
  An example from Figure 15.1 of ``Book 1,'' is
  $\{(01,1), (011,1), (01,0), (1,101)\}$. \\
  This can be interpreted as detailed below:
  \begin{itemize}
  \item[]
    $\{(\alpha_1 = 01, \beta_1 = 1),$
  \item[] $ (\alpha_2 = 011, \beta_ 2 = 1),$
   \item[] $ (\alpha_3 = 01,  \beta_3 = 0),$
   \item[]  $ (\alpha_4 = 1,   \beta_4 = 101)\}$.
  \end{itemize}
  
\item \verb|[ ]| A solution for a PCP instance S is a sequence
  ${i1}, {i2}, \ldots, {in}$ such that \\
  $\alpha_{i1} \alpha_{i2} \ldots \alpha_{in} =
   \beta_{i1} \beta_{i2} \ldots \beta_{in}$. \\
  Continuing the example from the book, we have ``$4,3,1,4,2$'' as the
  solution sequence, as shown by \\
  $\alpha_{4} \;  \alpha_{3}  \; \alpha_{1} \; \alpha_{4} \; \alpha_{2} =
  \beta_{4} \; \beta_{3}  \; \beta_{1} \; \beta_{4} \; \beta_{2}$, which
  can be read as  \\
  $1\; 01\; 0    1 \; 1\; 01   1 =
   1   01\; 0 \; 1\;  1   01\; 1$.  

 \item Given a PCP instance $S$ as in Step~\ref{step1}, we generate a
   formula $W_S$ of the form
     \( (A_1 \wedge A_2) \Rightarrow C \)

   where

   \begin{itemize}
   \item[] $A_1 = 
\textstyle \wedge_{i=1}^{n} \; p(f_{\alpha_i}(a), f_{\beta_i}(a)) \;\;\;\;\;\;\;\;  $ 
\item[] $A_2 = 
\textstyle \forall x\; \forall y\;
  [p(x,y) \Rightarrow \wedge_{i=1}^{n} \; p(f_{\alpha_i}(x), f_{\beta_i}(y))] \;\;\;\;\;\;\;\; $ 
\item[] $C = 
\textstyle \exists z\; p(z,z) \;\;\;\;\;\;\;\; $ 
   \end{itemize}



 \item \verb|[ ]| First, we will try to prove this -- call it {\bf Case-1}:\\
   {\bf $W_S$ is valid implies S has a solution.}

 \item \verb|[ ]| Next, we will try to prove this -- call it {\bf Case-2}:\\
   {\bf S has a solution implies $W_S$ is valid.}

 \item \verb|[ ]| For {\bf Case-1}, assume $W_S$ is valid. That
   means we can pick any interpretation we like (to achieve our goal,
   which is $S$ has a solution). So pick the following interpretation.

\begin{list}{$\bullet$}{\addtolength{\itemsep}{-1.6\itemsep}}
\item[] $\displaystyle a = \varepsilon$
\item[] $\displaystyle f_0(x) = x0$ (string `$x$' and string `$0$' concatenated)
\item[] $\displaystyle f_1(x) = x1$ (similar to the above)
\item[] $\displaystyle p(x,y)$ = There exists a non-empty
                                sequence $i_1 i_2 \ldots i_m$ such that
\item[] $\;\;\;\;\;\;\;\;\;\;\;\;\;\;\;\;\;\;\;  x = \alpha_{i_1}\alpha_{i_2}\ldots\alpha_{i_m}$ and
                                $y = \beta_{i_1}\beta_{i_2}\ldots\beta_{i_m}$ 
\end{list}
%--
   

\item \verb|[ ]| Under this interpretation, we can see that each
  conjunct of $A_1$ is true. So $A_1$ as a whole is true. This is
  because $p$ says that ``it must see one block being
  fed to it.''

\item \verb|[ ]| Under this interpretation, we can see that inside
  the $\forall$, if we keep $x$ and $y$ general, then
  we have to decide if the following is true:
  \begin{itemize}
    \item $[p(x,y) \Rightarrow \wedge_{i=1}^{n} \; p(f_{\alpha_i}(x), f_{\beta_i}(y))] $
  \end{itemize}
  But first let's stare at this, for some $i$:
  \begin{itemize}
  \item $[p(x,y) \Rightarrow  \; p(f_{\alpha_i}(x), f_{\beta_i}(y))] $
  \end{itemize}

\item \verb|[ ]|
  Now,   $[p(x,y) \Rightarrow  \; p(f_{\alpha_i}(x), f_{\beta_i}(y))] $ is true for some $i$. This is because ``p says'' the following: if $x,y$
  are as per a consistent $\alpha$ and $\beta$ concatenation,
  we can extend it by one more $\alpha$ or $\beta$ concatenation.
  That is, we are making a longer listing similar to \\
  $\alpha_{4} \;  \alpha_{3}  \; \alpha_{1} \; \alpha_{4} \; \alpha_{2}$\\
  and\\
  $\beta_{4} \; \beta_{3}  \; \beta_{1} \; \beta_{4} \; \beta_{2}$.


\item \verb|[ ]|
  Thus, $A_2$ as a whole is true, where $A_2$ is as follows:\\
$\textstyle \forall x\; \forall y\;
  [p(x,y) \Rightarrow \wedge_{i=1}^{n} \; p(f_{\alpha_i}(x), f_{\beta_i}(y))] \;\;\;\;\;\;\;\; $ 
  

\item \verb|[ ]| So we know that  $A_1$ and $A_2$ are true. Also
  we know that $(A_1\wedge A_2\Rightarrow C)$ is true because
  we are working under {\bf Case-1} which assumes $W_S$ is valid.


\item \verb|[ ]| Thus $C$ must be true!

\item \verb|[ ]| This means that the PCP system $S$ has a solution! That is
  what $C$ says!  {\bf So, Case-1 is proven!}

\item \verb|[ ]|
  Next, we will try to prove  {\bf Case-2}:\\
   {\bf S has a solution implies $W_S$ is valid.}

\item \verb|[ ]|   
  Since $S$ has a solution, there is a sequence
  ${i1}, {i2}, \ldots, {in}$ such that \\
  $\alpha_{i1} \alpha_{i2} \ldots \alpha_{in} =
   \beta_{i1} \beta_{i2} \ldots \beta_{in}$.


 \item \verb|[ ]| We must show $W_S$ is valid -- true under
   all interpretations!

 \item \verb|[ ]|   
   So we can't assume any particular meaning
   for $p$, $f_0$, and $f_1$. Remember that things like 
   $f_{\alpha_i}$, $f_{1011}$, etc are abbreviations. We really
   only have $f_0$ and $f_1$ as our function symbols.

 \item \verb|[ ]|   But $W_S$ is of the form
   $(A_1 \wedge A_2 \Rightarrow C)$ and we are asked to show
   it is valid!

 \item \verb|[ ]| It is silly to consider the case
   of $A_1$ or $A_2$ being false -- because then $W_S$ would
   be true. Thus we can assume that $A_1$ and $A_2$ are true,
   and show that $C$ is true -- under all interpretations!

\item    Then we would have shown that
   $(A_1 \wedge A_2 \Rightarrow C)$ is valid.

\item \verb|[ ]| But we know that $S$ has a solution. So we exploit that.

\item \verb|[ ]|  
  Consider
  $A_1$. {\bf We are  assuming $A_1$ is true, and so we exploit that fact.}
  We can take $A_1$ and see that it asserts something
     rather simple. It asserts $n$ facts of this form:
   $ 
\textstyle \wedge_{i=1}^{n} \; p(f_{\alpha_i}(a), f_{\beta_i}(a)) \;\;\;\;\;\;\;\;  $      

\item \verb|[ ]| \label{step-x}
  Take {\bf one of these facts asserted by $A_1$.}
  And in fact,
  choose the {\em particular fact} following the solution $S$: \\
   $ 
\textstyle \; p(f_{\alpha_{i_1}}(a), f_{\beta_{i_1}}(a)) \;\;\;\;\;\;\;\;  $      
  

 \item \verb|[ ]| Fine. Have a coffee or beer before you continue!

 \item \verb|[ ]| Now, {\bf we are assuming that $A_2$ is true.}
   
 \item \verb|[ ]| We can interpret $A_2$ as a ``bag of inference
   rules.''
   
 \item \verb|[ ]| \label{step-y}
   $A_2 =$
$\textstyle 
   [p(x,y) \Rightarrow \wedge_{i=1}^{n} \; p(f_{\alpha_i}(x), f_{\beta_i}(y))] \;\;\;\;\;\;\;\; $

 \item \verb|[ ]|      Notice that in the above step,
   we can infer one big conjunct. We choose to infer one
   of those conjuncts only!
   
   
\item \verb|[ ]| Thus, we can ``pattern match'' the fact in
  Step~\ref{step-x} with the rule in   Step~\ref{step-y}
  and infer, using modus ponens, the following:

  \[ p(f_{\alpha_{i_2}} ( f_{\alpha_{i_1}} ( a)),
       f_{\beta_{i_2}} ( f_{\beta_{i_1}} ( a)) ) \]
  
  
  
     \item \verb|[ ]| Wow, this game is fun and we can keep inferring
       bigger and bigger things! We infer next, the following:


       \[ p(f_{\alpha_{i_3}} ( f_{\alpha_{i_2}} ( f_{\alpha_{i_1}} ( a))),
          p(f_{\beta_{i_3}} ( f_{\beta_{i_2}} ( f_{\beta_{i_1}} ( a)))) \]
       
        \item \verb|[ ]| Till finally we infer this:
\[ p(f_{\alpha_{i_1}\alpha_{i_2} \ldots\alpha_{i_m}}(a),
     f_{\beta_{i_1}\beta_{i_2} \ldots\beta_{i_m}}(a)). \]          

   \item \verb|[ ]|
     But we now know that
     \[ \alpha_{i_1}\alpha_{i_2}\ldots\alpha_{i_m} = \beta_{i_1}\beta_{i_2}\ldots\beta_{i_m} = Soln, \] because ``$S$ has a solution'' -- call it $Soln$.

   \item Thus  we inferred
     \( p(f_{Soln}(a), f_{Soln}(a)) \).

   \item \verb|[ ]|
     This means that we inferred $C$ is true !!


   \item \verb|[ ]|
     Thus, without assuming any interpretation at all,
     assuming $A_1$ and $A_2$, we showed that $C$ is true.

   \item \verb|[ ]|   Thus we showed that $W_S$ is valid. This
     finishes {\bf Case-2}.
  
\end{enumerate}

\end{Large}

\end{document}
%==

