
% --------------------------------------------------------------
% This is all preamble stuff that you don't have to worry about.
% Head down to where it says "Start here"
% --------------------------------------------------------------

\documentclass[11pt]{article}

\usepackage{bera}
%\renewcommand{\familydefault}{\rmfamily}

\usepackage{graphicx,url}
\usepackage{proof}
\usepackage{framed}
\usepackage{etaremune}

\usepackage[margin=1in]{geometry}
\usepackage{amsmath,amsthm,amssymb,amsfonts}
\usepackage{paralist}
\thispagestyle{empty}

% 1. To get version suitable for students to populate,
%    remove the contents of the \ignoreSoln{..body..}
%
% 2. To get a version suitable for generating PDF 
%    without solutions, remove the #1 below
%
% 3. To generate solutions, keep the #1 below
%
% 4. Assigned grader fills \ignoreSoln{..body..}
%    and also provides his/her feedback to student
%    and policy followed for point deduction
%    So design policy before grading begins.

\newcommand{\ignoreSoln}[1]{#1}   
%\newcommand{\ignoreModel}[1]{#1} 


\newcommand{\bigset}[2]{\big\{\;#1\;:\;#2\;\big\}}
\newcommand{\N}{\mathbb{N}}
\newcommand{\Z}{\mathbb{Z}}
\newcommand{\R}{\mathbb{R}}
\newcommand{\Np}{\mathbb{N^{+}}}

\newenvironment{theorem}[2][Theorem]{\begin{trivlist}
\item[\hskip \labelsep {\bfseries #1}\hskip \labelsep {\bfseries #2.}]}{\end{trivlist}}
\newenvironment{lemma}[2][Lemma]{\begin{trivlist}
\item[\hskip \labelsep {\bfseries #1}\hskip \labelsep {\bfseries #2.}]}{\end{trivlist}}
\newenvironment{exercise}[2][Exercise]{\begin{trivlist}
\item[\hskip \labelsep {\bfseries #1}\hskip \labelsep {\bfseries #2.}]}{\end{trivlist}}
\newenvironment{reflection}[2][Reflection]{\begin{trivlist}
\item[\hskip \labelsep {\bfseries #1}\hskip \labelsep {\bfseries #2.}]}{\end{trivlist}}
\newenvironment{proposition}[2][Proposition]{\begin{trivlist}
\item[\hskip \labelsep {\bfseries #1}\hskip \labelsep {\bfseries #2.}]}{\end{trivlist}}
\newenvironment{corollary}[2][Corollary]{\begin{trivlist}
\item[\hskip \labelsep {\bfseries #1}\hskip \labelsep {\bfseries #2.}]}{\end{trivlist}}

\DeclareMathSizes{14}{14}{14}{14}

\begin{document}

% --------------------------------------------------------------
%                         Start here
% --------------------------------------------------------------

%\renewcommand{\qedsymbol}{\filledbox}


\begin{center}
\begin{large}
  CS 6110, Spring 2022, Assignment 2  \\
  Given 1/25/22 -- Due 2/1/22 by 11:59 pm via your Github 
  \ \\
%  \ \\  
    {  {\Large\bf NAME: } \hfill {\Large\bf UNID: }\hspace{4cm} }
          \ \\
\end{large}

\end{center}

\noindent{\bf CHANGES:\/} {\bf Please look for lines beginning with underlined words.}

\noindent {\bf Answering, Submission:\/}
Please provide a PDF with these questions answered
in the spaces indicated (if you don't retain the frameboxes,
at least please begin each answer on a new page
with the question numbers/parts indicated).
Also have your work ready on your Github for me to pull/test.
%
Note that Asg2 will be gone over in class and even partially
worked out; you'll be finishing the unfinished parts and submitting
the full solution.
%
Orientation videos and further help will be available (drop a note anytime
on Piazza for help)---{\bf start early}.

\begin{enumerate}
  
%- 1 ----------------------------------------------------------------
\item (20 points) Read about LTL from 
  CEATL's Chapter 22.
  (You may also look at Ben-Ari's Chapter 5.)
  By way of practice,
  in the directory {\tt Lec5}
  within the class Git
  \url{https://github.com/ganeshutah/cs6110s22.git},
  you have been given four Promela files {\tt p1.pml}
  through {\tt p4.pml}.   
  %
  Run these for your own understanding and repeat the results
  at the end of these files (if any).
  %
  In particular, in {\tt p4.pml}, you are checking
  whether {\em Justice} implies {\em Compassion}.
  %
  These terms are defined in the paper
  ``All you need is Compassion''
  \url{http://citeseerx.ist.psu.edu/viewdoc/summary?doi=10.1.1.124.3239}
  (more properly \url{https://dl.acm.org/doi/10.5555/1787526.1787547}).
  %
  Does Justice imply Compassion?
  %
  Does Compassion imply Justice?
  %
  Based on your observations (please provide terminal sessions \underline{showing
  the error trace(s)} that substantiate your conclusions),
  do you agree that Pnueli's paper's title is true?
  %
  Also answer these:
  \begin{enumerate}
  \item A solicitor walks around the neighborhood knocking on one door,
    then the other, taking a ``round-robin'' walk, and the home-owners
    are found to be infinitely-often opening and closing their doors
    in response.
    Are the home-owners being just of compassionate? Why?
    (\underline{justice} is also known as {\em weak fairness}, while
    \underline{compassionate} is also known as {\em strong fairness}).

  \item A solicitor walks to one door, leans on the bell and rings
    the bell continuously (till it almost melts).
    The home-owner infinitely-often opens the door.
    Now are the home-owners being just of compassionate?     Why?

  \item Repeat the above answers assuming that the solicitor is
    a network packet and home-owner doors are ports of a switch.
  \end{enumerate}


\newlength{\minpagw}
\settowidth{\minpagw}{\hspace{40em}}

\begin{minipage}{\minpagw}
  \fbox{%
    \parbox{\linewidth}{%
      Your

      Answer

      Here
    }%
  }%
\end{minipage}

\clearpage
  

%- 2 ----------------------------------------------------------------

\item (20 points) 
  %
  Now turn to Page 419 of CEATL and look at questions 22.3 and 22.5.
  %
  In 22.5, a Promela model and {\tt never} automata
  that distinguish 22.3(a) and 22.3(b) are given.
  \begin{enumerate}
  \item Remove the {\tt never} and replace it with an equivalent LTL.
    For instance, in the code we have {\tt never !(foo)} and here,
    the LTL being checked is {\tt foo}.
    Make sure that my claimed checks work (the first formula is true
    of {\tt sb} but not {\tt sa}, while the second formula is true
    of both structures; notice that each diagram
    that warrants being a Kripke structure has been given the name
    {\tt sa} through {\tt sh}).

  \item In the earlier question,
    a method to check the validity of LTL
    formulas using the most general Kripke Structure was introduced.
    %
    Using that method, answer these questions, with evidence furnished.

\noindent{\bf We will call this ``LTL of sa'':}

\noindent{\tt !(<>([](a \&\& b))) -> ((<>([]a)) || (<> (!a \&\& !b)))  }

\noindent{This can be read as ``[Not eventually-henceforth (a and b)]
  IMPLIES [either (eventually-henceforth a)
    or
    eventually (!a and !b)]''}
 
\noindent{We call this ``LTL of sa'' because on Page 420, we have this
  (in a never automation,
  the LTL is given one more negation):}

\begin{verbatim}
/* sb satisfies this, and not sa */
/*Type `spin -f "formula"' and cut&paste the resulting never aut. */
/*----------------------------------------------------------------*/
never {/* !( !(<>([](a && b))) -> ((<>([]a)) || (<> (!a && !b)))) */
\end{verbatim}


\noindent{\bf We will call this ``LTL of sb'':}

\noindent{\tt !(<>([](a \&\& b))) -> ( (<>([]a)) || (<> (b)) ) }

\noindent{This can be read as ``[Not eventually-henceforth (a and b)]
  IMPLIES [either (eventually-henceforth a)
    or
    eventually b]''}

\noindent{This is because on Page 421, we have this (in a never automation,
   the LTL is given one more negation):}

\begin{verbatim}
/*--- in contrast, both sa and sb satisfy this --->
 never {  /* !( !(<>([](a && b))) -> ( (<>([]a)) || (<> (b)) ) ) * /}
 <---*/
\end{verbatim}

    
    %
    \begin{enumerate}
    \item    Does the LTL of {\tt sa} implies that of {\tt sb}?
      Provide reasons after observing the trace from your experiment.


      
    \item    Does the LTL of {\tt sb} implies that of {\tt sa}?
      Provide reasons after observing the trace from your experiment.

   
    \end{enumerate}
  \end{enumerate}

% From: https://darrengoossens.wordpress.com/2019/09/01/latex-box-several-paragraphs-of-text-tightly/
% Set width of minipage (minpagw) to length of longest line by using
% the line as as argument to \settowidth


%\newlength{\minpagw}
%\settowidth{\minpagw}{\hspace{40em}}

\begin{minipage}{\minpagw}
  \fbox{%
    \parbox{\linewidth}{%
      Your

      Answer

      Here
    }%
  }%
\end{minipage}

\clearpage

%- 3 ----------------------------------------------------------------

\item (20 points) Fix the broken Bubble-sort, either through
  a mild repair or a wholesale rewrite.
  Verify (upto the limits of finite-state model checking) that
  it works.
  \underline{Submit evidence} that you ran exhaustively (possible for this simple a model).
    For fun, increase the array size from its current value in steps of 2 (or 1) and do a plot
    of the number of states generated, revisited, and matched. This tells you how many
    states are typically generated and how these grow. In symbolic (SAT/SMT-based model-checking),
    the states are not in a hash-table and the representation grows differently. In fact, a
    BDD-based hash-table has a size of $1$ when it is empty as well as when it is full!
    These are K-layer DFAs basically. See \url{https://spinroot.com/gerard/pdf/sttt98.pdf}.
  
\begin{minipage}{\minpagw}
  \fbox{%
    \parbox{\linewidth}{%
      Your

      Answer

      Here
    }%
  }%
\end{minipage}

\clearpage

%- 4 ----------------------------------------------------------------
\item (20 points) In the Philosophers example, the example suffered
  from
  ``livelock'' (all of them take the left, get nacked for the right,
  and repeat).
  One way to avoid such livelocks is to break symmetry (make one philosopher
  reach for the right fork first).
  %
  Implement this solution.
  %
  Now show that the system is lock-free but not wait-free
  (see \url{https://en.wikipedia.org/wiki/Non-blocking_algorithm}
  for these definitions).
  %
  Lock-free is {\em communal progress}
  and
  wait-free is {\em individual progress}.
  %
  \begin{compactitem}
  \item old instructions:
    \begin{scriptsize}
      \begin{compactitem}
      \item  Use LTL assertions (not {\tt never} automata).

      \item  Submit your work
      \end{compactitem}
  \underline{accompanied by traces and helpful observations}.
  \end{scriptsize}

\item new instructions:
  \begin{compactitem}
  \item Given the fix I posed on piazza (see {\tt dp\_contrarian.pml} pushed in)
  \item Understand the fix (the LTL used). Try it to make sure it works. Make any suggestions on improvement
  \end{compactitem}
  \end{compactitem}
  
\begin{minipage}{\minpagw}
  \fbox{%
    \parbox{\linewidth}{%
      Your

      Answer

      Here
    }%
  }%
\end{minipage}

\clearpage

%- 5 ----------------------------------------------------------------
\item (20 points) Follow-along and finish the design of
  Dijkstra's distributed termination algorithm.
  You'll be asked to type these with me in class live, and
  finish.
  %
  \url{http://people.cs.aau.dk/~adavid/teaching/MVP-10/17-Termination-lect14.pdf}
  and
  \url{https://www.cs.rochester.edu/u/sree/courses/csc-258/spring-2018/slides/22-td.pdf}
  do a good job of providing slides
  that I'll go thru in class.
  %
  The original is
  \url{https://www.cs.utexas.edu/~EWD/ewd08xx/EWD840.PDF}
  from Dijkstra's collection
  \url{https://www.cs.utexas.edu/~EWD/indexBibTeX.html}.
  %
  Look for files in the directory {\tt Lec5.}
  %
  \underline{Come Thu to type this} fully and check it!

  \underline{\bf ADDENDUM:\/}
  \begin{enumerate}
  \item {\tt DT.pml} is the initial template we began with
  \item  {\tt DT\_latest.pml} is the latest version we ended up with
  \item Your task is:
    \begin{enumerate}
    \item Begin with {\tt DT\_latest.pml} or your latest good solution
    \item Attempt a random initialization of the A/P status (see code below).
      While the original Dijkstra implementation started with the root alone
      in ``A,'' with the fix suggested below, all initializations are OK---which
      is a nice generalization!

    \item {\bf Fixed problem:\/} If you did not get any errors, then no fix needed! To tell you
      the back-story, I ran into a fix, then I thought the ``upstream logic'' was wrong.
      But now I can't reproduce the error. So perhaps this is after all a fixed protocol.
\begin{scriptsize}    
    \item Fix enough about the protocol's implementation (see Piazza discussion).
      The hint is to think about how ``upstream'' must be implemented.
      {\bf      As if you want more hints, but as a private Piazza post!}
    \item Then run the protocol verification!
\end{scriptsize}

    \end{enumerate}
  \end{enumerate}

  \noindent{\bf Detail of random initialization:\/}
  \begin{scriptsize}  
\begin{verbatim}
init {
byte i = Ns-1;
        atomic {
        do
        :: i > 0 ->     
           run node(tokqArray[i], tokqArray[i-1], workqArray[i], i);
	   if //-- nondet initialization of initial state
	   :: ns[i] = A 
	   :: ns[i] = P
	   fi;
           i-- ;	   
	   
        :: i == 0 ->
	   run node(tokqArray[0], tokqArray[Ns-1], workqArray[i], i);
	   if //-- nondet initialization of initial state
	   :: ns[i] = A
	   :: ns[i] = P
	   fi;	   
	   break
        od
        }
}
\end{verbatim}
  \end{scriptsize}
  

\begin{minipage}{\minpagw}
  \fbox{%
    \parbox{\linewidth}{%
      Your

      Answer

      Here
    }%
  }%
\end{minipage}

% --------------------------------------------------------------
\end{enumerate}

\end{document}
