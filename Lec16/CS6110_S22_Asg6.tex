
% --------------------------------------------------------------
% This is all preamble stuff that you don't have to worry about.
% Head down to where it says "Start here"
% --------------------------------------------------------------

\documentclass[11pt]{article}

\usepackage{bera}
%\renewcommand{\familydefault}{\rmfamily}

\usepackage{graphicx,url}
\usepackage{proof}
\usepackage{framed}
\usepackage{etaremune}

\usepackage[margin=1in]{geometry}
\usepackage{amsmath,amsthm,amssymb,amsfonts}
\usepackage{paralist}
\thispagestyle{empty}

% 1. To get version suitable for students to populate,
%    remove the contents of the \ignoreSoln{..body..}
%
% 2. To get a version suitable for generating PDF 
%    without solutions, remove the #1 below
%
% 3. To generate solutions, keep the #1 below
%
% 4. Assigned grader fills \ignoreSoln{..body..}
%    and also provides his/her feedback to student
%    and policy followed for point deduction
%    So design policy before grading begins.

\newcommand{\ignoreSoln}[1]{#1}   
%\newcommand{\ignoreModel}[1]{#1} 


\newcommand{\bigset}[2]{\big\{\;#1\;:\;#2\;\big\}}
\newcommand{\N}{\mathbb{N}}
\newcommand{\Z}{\mathbb{Z}}
\newcommand{\R}{\mathbb{R}}
\newcommand{\Np}{\mathbb{N^{+}}}

\newenvironment{theorem}[2][Theorem]{\begin{trivlist}
\item[\hskip \labelsep {\bfseries #1}\hskip \labelsep {\bfseries #2.}]}{\end{trivlist}}
\newenvironment{lemma}[2][Lemma]{\begin{trivlist}
\item[\hskip \labelsep {\bfseries #1}\hskip \labelsep {\bfseries #2.}]}{\end{trivlist}}
\newenvironment{exercise}[2][Exercise]{\begin{trivlist}
\item[\hskip \labelsep {\bfseries #1}\hskip \labelsep {\bfseries #2.}]}{\end{trivlist}}
\newenvironment{reflection}[2][Reflection]{\begin{trivlist}
\item[\hskip \labelsep {\bfseries #1}\hskip \labelsep {\bfseries #2.}]}{\end{trivlist}}
\newenvironment{proposition}[2][Proposition]{\begin{trivlist}
\item[\hskip \labelsep {\bfseries #1}\hskip \labelsep {\bfseries #2.}]}{\end{trivlist}}
\newenvironment{corollary}[2][Corollary]{\begin{trivlist}
\item[\hskip \labelsep {\bfseries #1}\hskip \labelsep {\bfseries #2.}]}{\end{trivlist}}

\DeclareMathSizes{14}{14}{14}{14}

\begin{document}

% --------------------------------------------------------------
%                         Start here
% --------------------------------------------------------------

%\renewcommand{\qedsymbol}{\filledbox}
\newlength{\minpagw}
\settowidth{\minpagw}{\hspace{40em}}

\begin{center}
\begin{large}
  CS 6110, Spring 2022, Assignment 6  \\
  Given 3/4/22 -- Due 3/15/22 by 11:59 pm via your Github 
  \ \\
%  \ \\  
    {  {\Large\bf NAME: } \hfill {\Large\bf UNID: }\hspace{4cm} }
          \ \\
\end{large}

\end{center}

\noindent{\bf CHANGES:\/} {\bf Please look for lines beginning with underlined words when they are made.}
         {\tiny none yet.}

         \noindent {\bf Answering, Submission:\/}
         Have these on your private Github:
         a folder Asg6/ containing your submission, which in detail comprises:
         \begin{compactitem}
         \item A clear README.md describing your files.
         \item Files that you ran + documentation (can be integrated in one place).
         \item A high level summary of your cool findings + insights + learning -- briefly reported in
           a nicely bulletted fashion in your PDF submission.
         \end{compactitem}

         \noindent {\bf Start Early, Ask Often!}
Orientation videos and further help will be available (drop a note anytime
on Piazza for help).

I encourage students constructing answers jointly! {\em But that does not
mean copy solutions, but discuss the question plus surrounding issues.}

\begin{compactenum}

%- 1 ----------------------------------------------------------------
\item (40 points)
  I've checked in Logic[1-5].als. There are two parts to this assignment:
  \begin{compactenum}
  \item (20 points)
    \begin{compactenum}
    \item (10 points)
    Go through what I've provided in the above models, and for each
    file, produce a gist of the encodings I've done and a  two-line
    comment saying how the encoding works. (Save Logic4.als for the
    next part.)

    \item (10 points)
    Get into Logic4.als and make sure that {\tt f2} indeed
    is encoding a general two-ary function. To ``make sure,'' you must
    demonstrate a sufficient number of alloy tests/checks/runs and/or
    model-examination. Have at least 3-4 convincing demos of checks
    you came up with.
      
    \end{compactenum}
    
  \item (20 points)
    \begin{compactenum}
    \item (10 points) Lecture 16, Slide 4, Image of question 13(b):
      Encode this assertion similar to the encoding in Logic5.als
      and check for validity.
      
    \item (10 points) Lecture 16, Slide 4, Image of question 13(c):
      Encode this assertion similar to the encoding in Logic5.als
      and check for validity.      
    \end{compactenum}
  \end{compactenum}

  
\begin{minipage}{\minpagw}
  \fbox{%
    \parbox{\linewidth}{%
      Your

      Answer

      Here
    }%
  }%
\end{minipage}

%- 2 ----------------------------------------------------------------
\item (40 points)
  \begin{compactenum}
  \item (20 points)
    \begin{compactenum}
    \item (10 points)
      Go through Mike-Gordon-Slides.pdf and also read Mike Gordon's book,
      making notes about the proof rules of Hoare Logic presented there.
      Include assignment, if, for, while, precondition
      strengthening, and postcondition weakening.

    \item (10 points)
      Get Dafny and Verifast installed. Please report issues you faced.
      Say which platform. Put notes in the tools GDoc to help each other.
    \end{compactenum}
    
  \item (20 points)
    Run the first two Dafny exercises in Lec16.pdf and ask questions
    on Piazza. All details are given to you.
  \end{compactenum}


  
\begin{minipage}{\minpagw}
  \fbox{%
    \parbox{\linewidth}{%
      Your

      Answer

      Here
    }%
  }%
\end{minipage}

%- 3 ----------------------------------------------------------------
\item (20 points)
  Write a summary of your project along these lines, occupying two pages.
  %
  Note: I'm adding topics to the GDoc
  \url{bit.ly/CS6110-S22-Project-Suggestions}
  and please add details there (this is a writeable GDoc also) or even
  new topics are welcome to be recorded there. In any case, please
  profit from the ideas there.
  %
  Push it into your Github which can then be used to drive your project also.
  %
  \begin{compactitem}
  \item A project name (tentative names are OK), a topic,
    and about 10 lines (12pt font) on why it matters to
    someone studying SW verification.
    
  \item A description of the verification technologies that you'll learn
    by doing this project. (It could be a topic not yet covered in class
    such as static analysis.)

  \item A few diagrams and other details you like to add to give me a better
    idea of the work.

  \item If you'd like to have a project partner, plz note that.

  \item Assuming you have 70\% of the CS 6110 time available for your
    project, a brief timeline of how you'll deliver your working project
    by the last day of Spring.
  \end{compactitem}
  
\begin{minipage}{\minpagw}
  \fbox{%
    \parbox{\linewidth}{%
      Your

      Answer

      Here
    }%
  }%
\end{minipage}  
  
%- end ----------------------------------------------------------------  


\end{compactenum}

\end{document}


