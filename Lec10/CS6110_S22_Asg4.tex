
% --------------------------------------------------------------
% This is all preamble stuff that you don't have to worry about.
% Head down to where it says "Start here"
% --------------------------------------------------------------

\documentclass[11pt]{article}

\usepackage{bera}
%\renewcommand{\familydefault}{\rmfamily}

\usepackage{graphicx,url}
\usepackage{proof}
\usepackage{framed}
\usepackage{etaremune}

\usepackage[margin=1in]{geometry}
\usepackage{amsmath,amsthm,amssymb,amsfonts}
\usepackage{paralist}
\thispagestyle{empty}

% 1. To get version suitable for students to populate,
%    remove the contents of the \ignoreSoln{..body..}
%
% 2. To get a version suitable for generating PDF 
%    without solutions, remove the #1 below
%
% 3. To generate solutions, keep the #1 below
%
% 4. Assigned grader fills \ignoreSoln{..body..}
%    and also provides his/her feedback to student
%    and policy followed for point deduction
%    So design policy before grading begins.

\newcommand{\ignoreSoln}[1]{#1}   
%\newcommand{\ignoreModel}[1]{#1} 


\newcommand{\bigset}[2]{\big\{\;#1\;:\;#2\;\big\}}
\newcommand{\N}{\mathbb{N}}
\newcommand{\Z}{\mathbb{Z}}
\newcommand{\R}{\mathbb{R}}
\newcommand{\Np}{\mathbb{N^{+}}}

\newenvironment{theorem}[2][Theorem]{\begin{trivlist}
\item[\hskip \labelsep {\bfseries #1}\hskip \labelsep {\bfseries #2.}]}{\end{trivlist}}
\newenvironment{lemma}[2][Lemma]{\begin{trivlist}
\item[\hskip \labelsep {\bfseries #1}\hskip \labelsep {\bfseries #2.}]}{\end{trivlist}}
\newenvironment{exercise}[2][Exercise]{\begin{trivlist}
\item[\hskip \labelsep {\bfseries #1}\hskip \labelsep {\bfseries #2.}]}{\end{trivlist}}
\newenvironment{reflection}[2][Reflection]{\begin{trivlist}
\item[\hskip \labelsep {\bfseries #1}\hskip \labelsep {\bfseries #2.}]}{\end{trivlist}}
\newenvironment{proposition}[2][Proposition]{\begin{trivlist}
\item[\hskip \labelsep {\bfseries #1}\hskip \labelsep {\bfseries #2.}]}{\end{trivlist}}
\newenvironment{corollary}[2][Corollary]{\begin{trivlist}
\item[\hskip \labelsep {\bfseries #1}\hskip \labelsep {\bfseries #2.}]}{\end{trivlist}}

\DeclareMathSizes{14}{14}{14}{14}

\begin{document}

% --------------------------------------------------------------
%                         Start here
% --------------------------------------------------------------

%\renewcommand{\qedsymbol}{\filledbox}
\newlength{\minpagw}
\settowidth{\minpagw}{\hspace{40em}}

\begin{center}
\begin{large}
  CS 6110, Spring 2022, Assignment 4  \\
  Given 2/11/22 -- Due 2/18/22 by 11:59 pm via your Github 
  \ \\
%  \ \\  
    {  {\Large\bf NAME: } \hfill {\Large\bf UNID: }\hspace{4cm} }
          \ \\
\end{large}

\end{center}

\noindent{\bf CHANGES:\/} {\bf Please look for lines beginning with underlined words when they are made.}
         {\tiny none yet.}

         \noindent {\bf Answering, Submission:\/}
         Have these on your private Github:
         a folder Asg4/ containing your submission, which in detail comprises:
         \begin{compactitem}
         \item A clear README.md describing your files.
         \item Files that you ran + documentation (can be integrated in one place).
         \item A high level summary of your cool findings + insights + learning -- briefly reported in
           a nicely bulletted fashion in your PDF submission.
         \end{compactitem}

         \noindent {\bf Start Early, Ask Often!}
Orientation videos and further help will be available (drop a note anytime
on Piazza for help). {\em I encourage students constructing answers jointly!}

\begin{enumerate}
  
%- 1 ----------------------------------------------------------------
\item (50 points - 25 for pre and 25 for partial - Alloy)
  To learn Alloy, you can get a PDF copy of the {\bf older} edition of the book
  by Daniel Jackson called ``Software Abstractions.''
  %
  I've found a PDF by searching for the above.
  %
  Since this is a very old edition,
  this is probably OK.
  %
  Other tutorials are\footnote{There is one more fantastic web-based one (I sent it to TM John
  once, but forgot where it is.)}

  \url{https://haslab.github.io/formal-software-design/overview/index.html}
  \url{https://www.cs.montana.edu/courses/se422/currentLectures/AlloyIntro.pdf} and
  \url{https://alloytools.org/tutorials/online/}.
  %
  %
  Start reading through this book and also Roger Costello's slides
  on the class Github.
  
\item[]  We have to understand mathematical relations properly before
  we can use Alloy.
  %
  Read my chapter in CEATL on relations (the chapter featured
  in the tutorial I recorded).
  %
  This is CS 2100 material---so, leaving it for your self-study.

\item[]    Here are some experiments I ran to study preorders and partial orders.
\item[] TL;DR {\em The intersection of a preorder and its inverse is not an
  identity relation.}
\item[] TL;DR {\em The intersection of a partial order and its inverse is  an
   identity relation.}  
\item[] These are the conclusions to be drawn in the experiments to follow.
  
\item[]  Your task is to fill out
    the ellipsed portions (where I provide an English phrase
    for you to fill as \verb|<text>|) and answer the questions below
    (questions begin with ``Q:'' and comments by ``C:''):

    \begin{footnotesize}
\begin{verbatim}
-- C: We are defining "some old" relation 'pre'
-- C: nd slowly endowing it with the properties that make it a preorder
--
sig S { pre  :  set S } -- (1) C: This defines 'pre' as a binary relation over S
fact  { some pre }      -- (2) Q: Can you explain what this means? 
fact  { <state here that pre is reflexive>  } -- (4) Q: answer the question below
fact  { <state here that pre is transitive> } -- (5) Q: answer the question below
assert preAndPreinvIden
 { <FALSELY Assert that the intersection of pre and its inverse is identity.> }
  -- (6) Q: answer the question below
check preAndPreinvIden for exactly 3 S
  -- (7) Q: Report on the result of this check
run {} for exactly 4 S
  -- (8) Q: If the check in (7) fails, comment (7) and run this to diagnose why
\end{verbatim}
    \end{footnotesize}
    
    \begin{compactitem}
    \item[] (1) {\tt pre} is introduced as a ``plain old'' relation
    \item[] (2) Explain what this assertion does for {\tt pre}
    \item[] (3) At this juncture, ``run'' for ``{\tt exactly 4 S}''
      and show the models generated.
    \item[] (4) How did you endow 'pre' with the property that it is reflexive? Explain.
    \item[] (5) Explain how you endowed 'pre' to be transitive
    \item[] (6) How did you falsely assert that the intersection of pre and its inverse is identity?
    \item[] (7) Did this check pass? You can use a pull-down menu and run this check.
      If the check in (7) failed, look at the counterexample and explain why it failed.
    \item[] (8) Also run this ``run'' statement and see the instance
      generated. Does the instance generated provide another explanation as to why the
      above {\tt check} failed?
      
    \item[] (9) Now, define a {\em partial order} along the same lines as below.
      Instead of  the assertion {\tt preAndPreinvIden},
      define {\tt partAndPartinvIden},      where we changed ``pre''
      to ``part'' (partial order).
      Did this check pass? Justify the answer.
    \end{compactitem}

%- 2 ----------------------------------------------------------------
\item (10 points, Store Buffer)
  Run the store-buffer example created in Promela. Argue that
  it simulates the situation of the writes being buffered before
  it goes to memory (as in TSO which is
  described at \url{https://en.wikipedia.org/wiki/Memory_ordering} and
  elsewhere).
  %
  Observe the assert failure and explain the interleaving causing the
  bug. (The file in question is \verb|Peterson_tso.prm|.)
  
\begin{minipage}{\minpagw}
  \fbox{%
    \parbox{\linewidth}{%
      Your

      Answer

      Here
    }%
  }%
\end{minipage}

%- 3 ----------------------------------------------------------------
\item (10 points, The Java example with volatiles)
  Read-up on Java volatiles. Run the example \verb|VBad.java|.
  Insert volatiles selectively (just for req or just for ack).
  Does that correct the apparent hang? (I don't know the answer
  but thought you'd like to try.) To get the apparent hang,
  first you must leave out the volatile totally and get the
  hangs on your machine. {\em Then} try to add one volatile
  and see if you get a hang. Explain your observations. (Bound
  your empirical testing to say an hour.)

  \begin{minipage}{\minpagw}
  \fbox{%
    \parbox{\linewidth}{%
      Your

      Answer

      Here
    }%
  }%
  \end{minipage}


%- 4 ----------------------------------------------------------------
\item (10 points, the MWGC game)
  Write a pseudo-code for a DFS model-checker by modifying the BFS model-checker's
  pseudo-code here
  \url{https://www.cs.utah.edu/~kirby/Publications/Kirby-33.pdf}.
  %
  Do you now understand why a model-checker does not ``infinitely loop?''
  %
  Now, run the murphi model I wrote today (you should be able to recreate it from
  memory or the recording) and run it in DFS.
  %
  Does the error-trace (winning sequence) get longer?
  %
  Can you go after longer error traces (by not stopping at the first error)?
  %
  Try to produce a longer error trace than with DFS.
  %
  Describe that ``winnning sequence'' (error trace for the negated invariant).

  \begin{minipage}{\minpagw}
  \fbox{%
    \parbox{\linewidth}{%
      Your

      Answer

      Here
    }%
  }%
\end{minipage}  
  
%- 5 ----------------------------------------------------------------  
\item (20 points, DCL)
  Read Pugh's analysis of the Java Memory Model through the URL
  \url{https://www.cs.umd.edu/~pugh/java/memoryModel/DoubleCheckedLocking.html}
  and describe what weak memory ordering issues are discussed there?
  (This is a classic website that sowed the whole area of understanding
  weak memory models; Java was supposed to be a ``safe'' language, but
  with weak memory, you could export an object reference before
  initializing the object---thus leaking a secret that you did not
  wipe out.)
  %
\begin{minipage}{\minpagw}
  \fbox{%
    \parbox{\linewidth}{%
      Your

      Answer

      Here
    }%
  }%
\end{minipage}    
%- end ----------------------------------------------------------------  


\end{enumerate}

\end{document}
